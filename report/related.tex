\section{Related Work}
Mutation testing has been around since 1970`s. Though the first implementation of mutation testing was introduced by DeMillo et al\cite{demilo}. In their work, they make one important assumption that experienced developers either write a correct program or an almost correct one. In other words, it is highly probable that experienced programmer makes simple mistake, such as using addition operator instead of subtraction. Therefore they define a list of 22 simple mutation operators which is applied on the original program. 
Zeller et al\cite{zeller} create a more sophisticated framework for mutation testing called Javalanche. They focus on efficiency and usefulness of mutation framework. For better efficiency, they only execute those tests that actually execute the mutated code. As for improved inspection, they define equivalency among mutants. Two mutants are considered equal if they leave the program semantic unchanged. As a result, they only run tests on those mutants that are not equal. Their work on mutation testing also motivated the later work \cite{zeller2} by the same authors on generating test oracle using the same mutation testing methodology. 